%%def%:=

%:\begin{verbatim}
%:Usage instructions  -  ATM on Linux, release 0.78
%:-------------------------------------------------
%:
%:\end{verbatim}
%%beginskip

\documentclass[a4paper,11pt]{article}
\usepackage{url}

\def\meta#1{{\it #1\/}} % ... <blah> ...
\def\path#1{{\tt #1}}   % ... /foo/bar ...
\def\raw#1{{\tt #1}}    % ...  raw output  ...
\def\craw#1{{\tt #1}}   % ...  raw output  ...
\def\name#1{{\sf #1}}   % ... FooBar ...

%%def\\meta{([^{}]*)}=<$1>
%%def\\path{([^{}]*)}=$1
%%def\\url{([^{}]*)}=~$1~
%%def\\raw{([^{}]*)}=~$1~
%%cmd\\craw{([^{}]*)}=((($x = $1) =~ tr/a-z/A-Z/), $x)
%%def\\name{={
%%def\\em=
%%def\\footnotesize=

\newenvironment{command}{\def\[{$\bigl[$}\def\]{$\bigr]$}\def\|{$\big\vert$}%
  \parindent=-2em\advance\leftskip by -\parindent\vskip -\parskip~\par
  \begingroup\tt\textfont0=\font}{%
  ~\endgroup\par~\advance\hoffset by \parindent}
%%def\\begin{command}=\\raw{$SI$SI$SI$SI
%%def\\end{command}=$SO$SO$SO$SO}

\parindent=0pt
\parskip=4pt
\tolerance=9000

\title{ATM on Linux \\
  User's guide \\
  Release 0.78 (beta)}
\author{Werner Almesberger \\
  {\tt Werner.Almesberger@epfl.ch} \\
  \\
  Institute for computer Communications and Applications (ICA) \\
  EPFL, CH-1015 Lausanne, Switzerland}
 
\date{July 6, 2000}

\begin{document}
\maketitle
{
  \parskip=-1pt
  \setcounter{tocdepth}{2}
  \tableofcontents
}

%%endskip


For updates of ATM on Linux, please check the Web page at
\url{http://icawww1.epfl.ch/linux-atm/}

%%beginskip
\vskip1cm
\centerline{\bf WARNING}
%%endskip
%:\begin{verbatim}
%:         WARNING
%:\end{verbatim}

This is experimental software. There are known major bugs and certainly
even many more yet unknown problems. Internal and external interfaces are
far from being stable. In fact, they change daily. Use at your own risk.


%------------------------------------------------------------------------------


\section{Installation}

In order to install this package, you need
\begin{itemize}
  \item the package itself
    \url{ftp://icaftp.epfl.ch/pub/linux/atm/dist/atm-0.78.tar.gz}
  \item the Linux kernel, version 2.4.0-test3-pre4, e.g. from
    \url{ftp://ftp.kernel.org/pub/linux/kernel/v2.4/linux-2.4.0-test2.tar.bz2}
    and
    \url{ftp://ftp.kernel.org/pub/linux/kernel/testing/test3-pre4.bz2}
  \item Perl, version 4 or 5
  \item if you want memory debugging: MPR version 1.6, e.g. from
    \url{ftp://sunsite.unc.edu/pub/Linux/devel/lang/c/mpr-1.6.tar.gz}
\end{itemize}


\subsection{The source tree}

First, create a directory in which the \path{linux/} kernel source directory
and the ATM on Linux distribution directory (\path{atm/}) should be created,
and put all the files listed above there. Then extract the ATM on Linux
distribution:

\begin{verbatim}
tar xfz atm-0.78.tar.gz
\end{verbatim}

and the kernel source:

\begin{verbatim}
tar xfI linux-2.4.0-test2.tar.bz2
bzcat test3-pre4.bz2 | patch -p0 -s
\end{verbatim}

Finally, you can extract the ATM-related patches:

\begin{verbatim}
cd linux
patch -s -p1 <../atm/atm.patch
\end{verbatim}

The distribution installs into the following directories:

\begin{description}
  \item[\path{atm/}] Documentation in ASCII format, kernel patch, top-level
    \name{Makefile}, and distribution scripts
  \item[\path{atm/sigd/}]  UNI 3.0, UNI 3.1, and UNI 4.0 signaling demon:
    \name{atmsigd}
  \item[\path{atm/saal/}] Signaling AAL library (SSCOP, SSCF, and SAAL)
  \item[\path{atm/qgen/}] Q.2931-style message handling
  \item[\path{atm/ilmid/}] ILMI address registration demon: \name{ilmid}
  \item[\path{atm/maint/}] ATM maintenance programs: \name{atmaddr},
    \name{atmdiag}, \name{atmdump}, \name{atmloop}, \name{atmtcp},
    \name{enitune}, \name{esi}, \name{sonetdiag}, \name{saaldump}, and
    \name{zntune}
  \item[\path{atm/test/}] Test programs: \name{align}, \name{aping},
    \name{aread}, \name{awrite}, \name{br}, \name{bw}, \name{isp},
    \name{ttcp\_atm}, \name{window}
  \item[\path{atm/arpd/}] ATMARP tools and demon: \name{atmarp}, \name{atmarpd}
  \item[\path{atm/led/}] LAN Emulation demon: \name{zeppelin}
  \item[\path{atm/lane/}] LAN Emulation servers: \name{bus}, \name{lecs},
    \name{les}
  \item[\path{atm/mpoad/}] Multi-Protocol Over ATM demon: \name{mpcd}
  \item[\path{atm/debug/}] Debugging tools: \name{delay},
    \name{ed}, \name{encopy}, \name{endump}, \name{svctor}, \name{zndump},
    and \name{znth}
  \item[\path{atm/lib/}] Libraries for applications and demons
  \item[\path{atm/doc/}] Documentation in \LaTeX\ and conversion tools
  \item[\path{atm/man/}] Miscellaneous man pages
  \item[\path{atm/extra/}] Extra packages (\name{tc}, \name{tcpdump}, and
    \name{ans})
  \item[\path{atm/config/}] Configuration file examples
  \item[\path{atm/switch/}] Switch fabric control (under construction)
\end{description}


\subsection{Kernel configuration}

Now, change links (if necessary) and do the usual \raw{make config},
\raw{make menuconfig}, or \raw{make xconfig}. First, enable
\begin{verbatim}
Prompt for development and/or incomplete code/drivers
  (CONFIG_EXPERIMENTAL)
\end{verbatim}

You should find the following new options:

\begin{verbatim}
Asynchronous Transfer Mode (ATM, EXPERIMENTAL) (CONFIG_ATM)
  Use "new" skb structure (CONFIG_ATM_SKB)
  Classical IP over ATM (CONFIG_ATM_CLIP)
    Do NOT send ICMP if no neighbour (CONFIG_ATM_CLIP_NO_ICMP)
  LAN Emulation (LANE) support (CONFIG_ATM_LANE)
    Multi-Protocol Over ATM (MPOA) support (CONFIG_ATM_MPOA)
ATM over TCP (CONFIG_ATM_TCP)
Efficient Networks ENI155P (CONFIG_ATM_ENI)
  Enable extended debugging (CONFIG_ATM_ENI_DEBUG)
  Fine-tune burst settings (CONFIG_ATM_ENI_TUNE_BURST)
    Enable 16W TX bursts (discouraged) (CONFIG_ATM_ENI_BURST_TX_16W)
    Enable 8W TX bursts (recommended) (CONFIG_ATM_ENI_BURST_TX_8W)
    Enable 4W TX bursts (optional) (CONFIG_ATM_ENI_BURST_TX_4W)
    Enable 2W TX bursts (optional) (CONFIG_ATM_ENI_BURST_TX_2W)
    Enable 16W RX bursts (discouraged) (CONFIG_ATM_ENI_BURST_RX_16W)
    Enable 8W RX bursts (discouraged) (CONFIG_ATM_ENI_BURST_RX_8W)
    Enable 4W RX bursts (recommended) (CONFIG_ATM_ENI_BURST_RX_4W)
    Enable 2W RX bursts (optional) (CONFIG_ATM_ENI_BURST_RX_2W)
ZeitNet ZN1221/ZN1225 (CONFIG_ATM_ZATM)
  Enable extended debugging (CONFIG_ATM_ZATM_DEBUG)
  Enable usec resolution timestamps (CONFIG_ATM_ZATM_EXACT_TS)
IDT 77201 (NICStAR) (CONFIG_ATM_NICSTAR)
  Use suni PHY driver (155Mbps) (CONFIG_ATM_NICSTAR_USE_SUNI)
  Use IDT77015 PHY driver (25Mbps) (CONFIG_ATM_NICSTAR_USE_IDT77105)
Madge Ambassador (Collage PCI 155 Server) (CONFIG_ATM_AMBASSADOR)
  Enable debugging messages (CONFIG_ATM_AMBASSADOR_DEBUG)
Madge Horizon [Ultra] (Collage PCI 25 and Collage PCI 155 Client)
  Enable debugging messages (CONFIG_ATM_HORIZON_DEBUG)
Interphase ATM PCI x575/x525/x531 (CONFIG_ATM_IA)
  Enable debugging messages (CONFIG_ATM_IA_DEBUG)
\end{verbatim}
%after ATM
%  Enable single-copy (CONFIG_MMU_HACKS)
%    Extended debugging for single-copy (CONFIG_MMU_HACKS_DEBUG)
%after CLIP
%  Application REQUested IP over ATM (CONFIG_AREQUIPA)
%after usec
%Rolfs TI TNETA1570 (CONFIG_ATM_TNETA1570)
%  Enable extended debugging (CONFIG_ATM_TNETA1570_DEBUG)


%The ``MMU hacks'' add single-copy support for raw AAL5 on adapters whose
%driver supports this (currently only the ENI155p). Extended debugging should
%only be enabled when tracing specific problems, because it slows down the
%drivers, and it may introduce new race conditions.

%The TNETA1570 driver is for a board developed by Rolf Fiedler at TU Chemnitz,
%see also \url{ftp://ftp.infotech.tu-chemnitz.de/pub/linux-atm}.

The burst settings of the ENI driver can be fine-tuned. This may be necessary
if the default settings lead to buffer overruns in the PCI chipset. See the
on-line help on \verb"CONFIG_ATM_ENI_TUNE_BURST" for a detailed discussion
of the implications of changing the burst settings.

Note that the file \path{drivers/atm/nicstar.h} contains a few configurable
settings for the IDT 77201 driver.

Some drivers can also be used with certain compatible cards. The latest
information about compatible cards can be found at
\url{http://icawww1.epfl.ch/linux-atm/info.html}

Then build your kernel and reboot.


\subsection{Driver messages}

If you've configured the ENI155p-MF driver, you should see two lines like
these:
{\footnotesize
\begin{verbatim}
eni(itf 0): rev.0,base=0xff400000,irq=10,mem=512kB (00-20-EA-00-07-56)
eni(itf 0): FPGA,MMF
\end{verbatim}}
(512kB for the -C version, 2048kB for the -S version.)

If you've configured the ZN1221/ZN1225 driver, you will get something like
{\footnotesize
\begin{verbatim}
zatm(itf 0): rev.3,base=0xf800,irq=11,mem=128kB,MMF (00-20-D4-10-2A-80)
zatm(itf 0): uPD98401 0.5 at 30.024 MHz
zatm(itf 0): 16 shapers, 32 pools, 2048 RX, 3958 VCs
\end{verbatim}}
Note that your board needs to be at least at revision level 3 if you want
to use it in a Triton-based system.

Note that if you've configured only the ATM over TCP driver, there are no
messages at startup, because ATM over TCP devices are created later using
the \name{atmtcp} command.


\subsection{Memory debugging}

If you want to enable debugging for options for memory allocations, you
need to install MPR before compiling the ATM tools. Extract
\path{mpr-1.6.tar.gz} into a directory at the same level as \path{linux}
and \path{atm}. Then do:

\begin{verbatim}
cd mpr-1.6
./configure x86-linux
make
install -c -m 0644 libmpr.a /usr/lib
install -c -m 0755 mpr mprcc mprhi mprlk mprsz /usr/local/bin
install -c -m 0755 mprfl mprnm mprpc mprdem mprmap /usr/local/bin
\end{verbatim}

Detection of some general mis-use of \raw{malloc} and \raw{free} is
automatically performed if the program was compiled with MPR present.
Tracing of allocations is enabled by setting \raw{MPRPC} and \raw{MPRFI}.
See \path{mpr.doc} in the MPR distribution for details.

Only little run-time overhead is incurred if memory debugging is included,
but those environment variables are not set.


\subsection{ATM tools}

Now, as the final step, configure and build the ATM tools. Configuration is
only necessary if your switch uses UNI 3.1 or 4.0, or if it has certain bugs.
The configuration options are at the beginning of \path{atm/Rules.make}.

Then, build the ATM tools with:

\begin{verbatim}
cd ../atm
make depend
make
make install
\end{verbatim}

\verb"make install" will install executables in the directory
\path{/usr/local/bin} and \path{/usr/local/sbin}, respectively.
Libraries and header files are installed in
\path{/usr/lib} and \path{/usr/include}, respectively. Man pages are
installed in \path{/usr/local/man}.


\subsection{Extra packages}

Some programs are based on large packages that are already distributed
outside of the ATM context. For such packages, only patches are contained
in the ATM on Linux distribution. The complete packages can be obtained
either from the original source (described in \path{atm/extra/extra.html})
or from \url{ftp://icaftp.epfl.ch/pub/linux/atm/extra/}.

The packages are automatically downloaded, patched, and built by running
\raw{make \meta{package\_name}} in the \path{atm/extra/} directory
(requires that the \name{Lynx} Web browser is installed).

Currently, the following extra packages are available:
\begin{description}
  \item[\name{tcpdump}] dumps network traffic (enhanced for ATM)
  \item[\name{ans}] ATM name server (based on \name{named} 4.9.5)
\end{description}

% Building \name{tcpdump} requires that \name{csh} is installed.
Note that \name{text2atm} automatically uses ANS if available, so
\name{ans} only needs to be installed on systems providing
name server functionality or if ATM-aware maintenance tools
(\name{nslookup}, etc.) are needed.

A script \name{hosts2ans.pl} to convert a \path{/etc/hosts.atm} file to
ANS zone files is provided in \path{atm/extra/}. Its use is described at
the beginning of the file.


%------------------------------------------------------------------------------


\section{Device setup}

This section describes device-specific configuration operations, and general
diagnostic procedures at the ATM or SONET level. Please see the adapter
documentation for details on hardware installation and diagnosis.


\subsection{ATM over TCP setup}

If you have no real ATM hardware, you can still exercise the API by using
the ATM over TCP ``driver''. It emulates ATM devices which are directly
wired to remote devices (i.e. there is no VPI/VCI swapping).

To establish one (bidirectional) ``wire'', become root on both systems
(or run both sides on the same system to create two connected ``interfaces'')
and run the following command on one of them (let's call it ``a''):

\begin{command}
a\# atmtcp virtual listen
\end{command}

Then, on the other system (``b''), run

\begin{command}
b\# atmtcp virtual connect \meta{address\_of\_a}
\end{command}

Both \name{atmtcp}s will report on their progress and the kernel should
display messages like

\begin{verbatim}
Link 0: virtual interface 2
Link 1: incoming ATMTCP connection from 127.0.0.1
\end{verbatim}

and

\begin{verbatim}
Link 0: virtual interface 3
Link 1: ATMTCP connection to localhost
\end{verbatim}

on the two systems. Note that \name{atmtcp} keeps running and that interrupting
it breaks the virtual wire.

Multiple ``wires'' can be attached to the same machine by specifying a
port number (default is 2812). Note that no AAL processing is performed.
It is therefore not possible to receive data using a different AAL (e.g.
AAL0) than the one with which the data was sent.


\subsection{ZN1221/ZN1225 tuning}

The ZeitNet ZN1221 and ZN1225 adapters use pre-allocated pools of free
memory buffers
for receiving. Whenever a VC with a certain maximum SDU size is opened for
receiving, the corresponding pool is filled with free buffers by the device
driver. The adapter removes buffers while it receives data. When the number
of remaining buffers falls below a certain threshold, the device driver
replenishes the pool again.

The lower and the upper limits for the number of free buffers, and the
threshold for adapting to a new data offset (see below for details), can
be set using the \name{zntune} program. Usage:

\begin{command}
zntune \[-l \meta{low\_water}\] \[-h \meta{high\_water}\]
  \[-t \meta{threshold}\] \meta{itf} \[\meta{pool}\]
\end{command}

The changes are applied to all pools if no pool number is specified.
Pool 2 stores 64 bytes packets, pool 3 stores 128 bytes packets, etc.
Pools 0 and 1 are currently unused.

The current settings and some usage statistics can be obtained by invoking
\name{zntune} without specifying new parameters:

\begin{command}
zntune \[-z\] \meta{itf} \[\meta{pool}\]
\end{command}

The ``Size'' column shows the buffer size in Bytes.
The ``Ref'' column shows the number of open VCs using that pool. The ``Alarm''
column shows how many times the number of free buffers has fallen below the
low-water mark since the counters were reset. Similarly, the ``Under'' column
shows how many times an incoming PDU had to be discarded because the
corresponding pool was empty.

The columns ``Offs'', ``NxOf'', ``Count'' and ``Thres'' show the alignment
adaption status. ``Offs'' is the offset of user data the driver currently
expects in incoming PDUs. For single-copy, receive buffers are aligned
accordingly so that data is received at page boundaries. ``NxOf'' is the
user data offset of the most recently received PDU, where the offset differs
from the currently assumed offset. ``Count'' is the number of PDUs that have
been received in sequence with an offset of ``NxOf''. Finally, ``Thres'' is
the threshold value ``Count'' has to reach for ``NxOf'' to become the new
current offset.

Use the \raw{-z} option to reset the ``Alarm'' and ``Under'' counters.


\subsection{Files in \path{/proc/net/atm}}

Some status information about the ATM subsystem can be obtained through files
in \path{/proc/net/atm}. \path{/proc/net/atm/arp} contains information
specific to Classical IP over ATM, see section \ref{clip}.

\path{/proc/net/atm/devices} lists all active ATM devices. For each device,
the interface number, the type label, the end system identifier (ESI), and
statistics are shown. The statistics correspond to the ones available via
\name{atmdiag}.

Individual ATM devices may register entries of the form
\raw{\meta{type}:\meta{number}} (e.g. \raw{eni:0}) which contain
device-specific information.

\path{/proc/net/atm/pvc} and \path{/proc/net/atm/svc} list all PVC and SVC
sockets.
For both types of sockets, the interface, VPI and VCI numbers are shown. For
PVCs, this is followed by the AAL and the traffic class and the selected
PCR for the receive and the transmit direction. For SVCs, the SVC state
and the address of the remote party are shown. SVCs with the interface
number 999 are used for special control purposes as indicated in the ``State''
column.

Furthermore, \path{/proc/net/atm/vc} shows buffer sizes and additional
internal information for all ATM sockets.


\subsection{ATM diagnostics}

Various counters of the ATM device drivers can be queried with the
\name{atmdiag}  program. See the corresponding man page for details.


\subsection{SONET diagnostics}

The SONET diagnostics tool can be used to monitor link performance
and to simulate errors. In order to get current SONET statistics,
run it with the ATM interface number as the argument, e.g.

\begin{verbatim}
% sonetdiag 0
\end{verbatim}

The counters can be reset with the \raw{-z} option:

\begin{verbatim}
# sonetdiag -z 0
\end{verbatim}

The following network failures can be simulated:\footnote{Some adapters
may only support a subset of this.}

\begin{description}
  \item[\raw{sbip}] insert section errors (B1)
  \item[\raw{lbip}] insert line errors (B2)
  \item[\raw{pbip}] insert path errors (B3)
  \item[\raw{frame}] force (RX) frame loss
  \item[\raw{los}] insert loss of signal
  \item[\raw{lais}] insert line alarm indication signal
  \item[\raw{pais}] insert path alarm indication signal
  \item[\raw{hcs}] insert header checksum errors
\end{description}

A failure is enabled by adding the corresponding keyword on the
command line. The failure is cleared by prefixing the keyword with
a minus sign, e.g.

\begin{verbatim}
a# sonetdiag -z 0 >/dev/null
b# sonetdiag -z 0 >/dev/null
a# sonetdiag 0 los
a# sonetdiag 0 -los
b# sonetdiag 0 | grep BIP
Section BIP errors:      56200
Line BIP errors:           342
Path BIP errors:           152
a# sonetdiag 0 | grep FEBE
Line FEBE:                 342
Path FEBE:                 152
\end{verbatim}

If any diagnostic error insertions are active, their keywords are
shown when \name{sonetdiag} is used to obtain statistics. Note that some
error insertions may be automatically switched off by the hardware.


%------------------------------------------------------------------------------


\section{Native ATM PVCs}

PVCs can be used for machines that are either connected back to back or
via a switch. In the latter case, the cell forwarding has to be manually
set up at the switch.


\subsection{Traffic tools}

\name{aread}/\name{awrite} and \name{br}/\name{bw} are simple programs to
access the ATM API. \name{awrite} sends the text string passed as its
second argument in an AAL5 PDU. \name{aread} receives one AAL5 PDU and
displays it in hex. Both programs also display the return values of the
corresponding system calls and the current values of \name{errno}.

\name{bw} either sends its standard input or a stream of blocks containing
arbitrary data (if a number is passed as its fourth argument) in 8 kB
AAL5 PDUs. \name{br} receives AAL5 PDUs and writes them to standard output.

The first argument of \name{aread}, \name{awrite}, \name{br} and \name{bw}
is always the PVC address,
i.e. the ATM interface number, the VPI and the VCI number, with a dot
between elements. The interface number can be omitted if it is zero.
Example:

\begin{verbatim}
% awrite 1.0.42 hi
\end{verbatim}

Note that some adapters only support VPI == 0. Also, the VCI range may be
limited, e.g 0 to 1023.
The interface number can be obtained from the initialization
message the driver printed during startup. \name{atm0} is interface 0,
\name{atm1} is interface 1, etc. If the system is equipped with a real
ATM adapter (e.g. not only \name{atmtcp}), that adapter is normally at
\name{atm0}.

\name{aping} receives and sends small AAL5 PDUs on a PVC. It expects that
messages it sends are either echoed back or that a similar program on the
other side generates a stream of messages. \name{aping} reports an error
if no messages are received for too long. \name{aping} is invoked by
specifying the PVC, like \name{aread}.

For "real" tests, you should use the modified version of \name{ttcp} that
comes with this package. The original is on
\url{ftp://ftp.sgi.com/sgi/src/ttcp}. The following options have been added:

\begin{description}
  \item[\raw{-a}] use native ATM instead of UDP/TCP. The address must be in
    the format \raw{$[$\meta{itf}.$]$\meta{vpi}.\meta{vci}} for PVCs, or a
    valid ATM end system address for SVCs.
  \item[\raw{-P \meta{num}}] use a CBR connection with a peak cell rate of
    \meta{num} cells per second. Default is to use UBR.
  \item[\raw{-C}] disable (UDP) checksums
\end{description}

Example:
\begin{verbatim}
%a ttcp_atm -r -a -s 0.90
%b ttcp_atm -t -a -s 0.90
\end{verbatim}


\subsection{Direct cell access}

On adapters where the device driver supports access to raw cells (``AAL0''),
individual cells can be composed and received with the \name{atmdump} program.

Man page: \name{atmdump.8}

Example:
{\footnotesize
\begin{verbatim}
a% sleep 10; date | ./atmdump -t 1 -c 0.51
b% ./atmdump 0.51
825079645.192480: VPI=0 VCI=51, GFC=0x0, CLP=1, Data SDU 1 (PTI 1)
   46 72 69 20 46 65 62 20 32 33 20 31 32 3a 34 37 
   3a 32 35 20 47 4d 54 20 31 39 39 36 0a 00 00 00 
   00 00 00 00 00 00 00 00 00 00 00 00 00 00 00 00 
\end{verbatim}}


%------------------------------------------------------------------------------


\section{Signaling}

\subsection{ATM hosts file}
\label{hosts}

Because ATM addresses are inconvenient to use, most ATM tools also
accept names instead of numeric addresses. The mapping between names and
numbers is defined in the file \path{/etc/hosts.atm}. The structure of
this file is similar to the \path{/etc/hosts} file:

\meta{numeric\_address} \meta{name(s)}

e.g.

\begin{verbatim}
47.0005.80FFE1000000F21A26D8.0020EA000EE0.00 pc2-a.fqdn pc2-a
47.0005.80FFE1000000F21A26D8.0020D4102A80.00 pc3-a.fqdn pc3-a
\end{verbatim}

The numeric address can be specified in any of the formats described
in \cite{api}. The numeric address(es) of a Linux system can be
determined with the command \raw{atmaddr -n} (see also section \ref{atmaddr}).

Many ATM tools also attempt to find the corresponding name when displaying
an address. When translating from the numeric form to a name, the first
applicable name in the file is used.

In addition to ATM addresses for SVCs, also PVC addresses can be stored in
\path{/etc/hosts.atm}. If different address types are stored under the
same name, the first suitable one will be chosen, i.e. if an application
explicitly requests only SVC addresses, any PVC addresses will be ignored.


\subsection{ANS}

If you have access to the ATM Name Service (ANS, e.g because you've installed
the ANS extension), you can use it instead of or in addition to the hosts
file by specifying the host that runs ANS in the \path{/etc/resolv.conf}
file.

For performing reverse lookups of E.164 addresses, the list of telephony
country codes needs to be known. That list can be obtained from
\url{http://www.itu.ch/itudoc/itu-t/lists/} the file has a name of the
form \path{tf\_cc\_e\_*.rtf} The script \path{atm/lib/rtf2e164\_cc.pl} can
be used to create the E.164 county codes table, e.g.

\begin{verbatim}
perl rtf2e164_cc.pl <tf_cc_e_13130.rtf >/etc/e164_cc
\end{verbatim}


\subsection{Signaling demon}

Man pages: \name{atmsigd.8}, \name{atmsigd.conf.4}

Note that \name{atmsigd}'s support for point-to-multipoint is very limited:
only operation as a single leaf of a point-to-multipoint tree works.

By default, \name{atmsigd} is configured to conform to UNI 3.0. It can be
compiled for UNI 3.1 by changing the \raw{STANDARDS=} line at the beginning
of \path{atm/Rules.make}, and running \raw{make clean; make} in
\path{atm/qgen} and \path{atm/sigd} (in this order).

Note that \name{atmsigd} is configured to be paranoid. If it detects unusual
problems, it frequently terminates. This will (obviously) change in the
future.

\name{atmsigd} also looks for a configuration file at the location specified
with the \raw{-c} option. The default location is \path{/etc/atmsigd.conf}.


\subsection{ILMI demon}

ILMI provides a mechanism for automatic address configuration. If there is
no switch or if the switch doesn't support ILMI, the ATM addresses must
be configured manually (see section \ref{atmaddr}). Note that the ILMI
demon should not be used on interfaces where addresses are manually
configured.

The ILMI demon is started as follows:

\begin{command}
\# ilmid \[-b\] \[-d\] \[-i \meta{local\_ip}\] \[-l \meta{logfile}\]
  \[-u \meta{uni\_version}\] \[-v\] \[-x\] \[\meta{itf}\]
\end{command}

\begin{description}
  \item[\raw{-b}] background. Run in a forked child process after initializing.
  \item[\raw{-d}] enables debugging output. By default, \name{ilmid} is very
    quiet.
  \item[\raw{-i \meta{local\_ip}}] IP address to tell switch when asked for one.
    Can be in either dotted decimal or textual format. By default, \name{ilmid}
    uses some heuristics to select a local IP address.
  \item[\raw{-l \meta{logfile}}] write diagnostic messages to the specified
    file instead of to standard error. The special name \raw{syslog} is
    used to send diagnostics to the system logger.
  \item[\raw{-q \meta{qos}}] configures the ILMI VC to use the specified
    quality of service. By default, UBR at link speed is used on the ILMI VC.
  \item[\raw{-u \meta{uni\_version}}] set UNI version. Possible values are
    \raw{3.0}, \raw{3.1}, and \raw{4.0}. The dot can be omitted. The default
    value depends on how \name{ilmid} was compiled. Typically, it is \raw{3.0}.
  \item[\raw{-v}] enables extensive debugging output.
  \item[\raw{-x}] disable inclusion of variable bindings in the
    ColdstartTrap. Some switches (e.g. the LS100) only work if this option
    is set.
\end{description}

If no interface number is specified, \name{ilmid} serves interface 0.
You can check whether address registration was successful with the
\name{atmaddr} command (see below).

The agent supports only the address registration procedures specified
in section 5.8 of the ATM Forum's UNI 3.1 specification.  These
procedures involve the switch registering the network prefix on the
host and the host registering the final ATM address back on the
switch.  The host accomplishes this by appending an ESI (End System
Identifier) and a null selector byte to the network prefix registered
by the switch.  The ESI is the physical or MAC address of the ATM
interface.


\subsection{Manual address configuration}
\label{atmaddr}

If your switch doesn't support ILMI, you have to set the ATM address
manually on the switch and on the PC(s). On the Linux side, make sure that
\name{ilmid} doesn't interfere, then use the \name{atmaddr} command to set
the address(es).

Man pages: \name{atmaddr.8}

Manual configuration of ATM addresses on the switch depends on the brand.
On a Fore ASX-200, it can be done with the following command:

\begin{command}
conf nsap route new \meta{nsap\_addr} 152 \meta{port} \meta{vpi}
\end{command}

e.g.

{\footnotesize
\begin{verbatim}
conf nsap route new 47000580ffe1000000f21510650020ea000ee000 152 1a2 0
                    |<---- NSAP prefix ----->||<--ESI--->|^^
                                                          SEL
\end{verbatim}}

The entire NSAP address always has to have a length of 40 digits.
Note that you can also use addresses with a different prefix and an ESI
that doesn't correspond to any ESI your adapters have. The value of the
selector byte (SEL) is ignored.


\subsection{Running two ATM NICs back-to-back}

It is also possible to run with two ATM NICs connected back-to-back,
and no switch in between\footnote{This section was written by
Richard Jones \raw{rjones@imcl.com}. Comments should be directed to me as well
as to Werner.}. This is great for simple test environments.

First, if you're using UTP or STP-5, you need a suitable cable.  Our
experience with standard 100Base-T back-to-back cables was not
good. It appears that the pin-out they use is different. After some
false starts, we found that the following cable works:

\begin{verbatim}
RJ45                            RJ45
   1        ------------        7
   2        ------------        8

   7        ------------        1
   8        ------------        2

Pins 3, 4, 5, 6 unconnected.
\end{verbatim}

You can also make up a loopback cable with 1 -- 7 and 2 -- 8 connected for
ultra-cheap setups.

Here we have two machines called ``virgil'' and ``nestor''.
Substitute your own names as necessary.

One side of the ATM connection needs to use the network version of
\name{atmsigd} and the other side should use the normal user version.
So here on nestor we start \name{atmsigd} with:

\begin{verbatim}
atmsigd -b -m network
\end{verbatim}

and on virgil with:

\begin{verbatim}
atmsigd -b
\end{verbatim}

Without a switch, you won't be able to use ILMI. Instead, create a
\path{/etc/hosts.atm} file containing two dummy addresses.
Our ATM hosts file contains:

\begin{verbatim}
47.0005.80FFE1000000F21A26D8.0020EA000EE0.00    nestor-atm
47.0005.80FFE1000000F21A26D8.0020D4102A80.00    virgil-atm
\end{verbatim}

These are completely spurious addresses, of course, but as long as you're
not connected to a public or private ATM network, I don't think it matters.
To set the address correctly in the driver, we use:

\begin{verbatim}
atmaddr -a virgil-atm
\end{verbatim}

on virgil, and:

\begin{verbatim}
atmaddr -a nestor-atm
\end{verbatim}

on nestor. Now start \name{atmarpd} on both machines
in the normal way. Now you (should) have a working ATM set-up. To get
IP over ATM working, just follow the instructions in
section \ref{ipoveratm}.


\subsection{Q.2931 message dumper}

The Q.2931 message compiler also generates a pretty-printer for Q.2931
messages. The executable is called \name{q.dump} is stored in the
\path{qgen} directory. Note that it is not copied elsewhere by
\raw{make install}

\name{q.dump} expects a sequence of whitespace-separated hex bytes at standard
input and outputs the message structure if the message can be parsed.

Example:

\begin{verbatim}
% echo 09 03 80 00 05 5A 80 00 06 08 80 00 02 81 83 00 48 \
  00 00 08 | ./q.dump
_pdsc = 9 "Q.2931 user-network call/connection control message"
_cr_len = 3
call_ref = 8388613 (0x800005)
msg_type = 0x5a "RELEASE COMPLETE"
_ext = 1
_flag = 0 "instruction field not significant"
_action_ind = 0 "clear call"
msg_len = 6 (0x6)
  _ie_id = 0x08 "Cause"
    _ext = 1
    cause_cs = 0 "ITU-T standardized"
    _flag = 0 "instruction field not significant"
    _action_ind = 0 "clear call"
    _ie_len = 2 (0x2)
      _ext = 1
      location = 1 "private network serving the local user"
      _ext = 1
      cause = 3 "no route to destination"
\end{verbatim}


%------------------------------------------------------------------------------


\section{IP over ATM}
\label{ipoveratm}

IP over ATM is supported with Classical IP over ATM (CLIP, defined in
RFC1577 \cite{RFC1577}), LAN Emulation (LANE, defined in~\cite{lanev1}
and~\cite{lanev2}) and Multi-Protocol Over ATM (MPOA, client only,
defined in~\cite{mpoav1}).


\subsection{CLIP}
\label{clip}

A demon process is used to generate and answer ARP queries.
The actual kernel part maintains a small lookup table only containing partial
information.

Man pages: \name{atmarpd.8}, \name{atmarp.8}

\name{atmsigd} and \name{ilmid} must already be running when \name{atmarpd} is
started. Use the \raw{-b} option to make sure they're properly synchronized,
e.g.

\begin{verbatim}
#!/bin/sh
atmsigd -b
ilmid -b
atmarpd -b
...
\end{verbatim}

works, but

\begin{verbatim}
#!/bin/sh
atmsigd &
ilmid &
atmarpd &
...
\end{verbatim}

frequently doesn't (yet).

The \name{atmarp} program is used to configure ATMARP. First, you have to
start \name{atmsigd}, \name{ilmid}, and \name{atmarpd}, then create an IP
interface and configure it:

\begin{command}
\# atmarp -c \meta{interface\_name}\\
\# ifconfig atm0 \meta{local\_address} \meta{possibly\_more\_options} up\\
\end{command}

e.g.

\begin{verbatim}
# atmarp -c atm0
# ifconfig atm0 10.0.0.3 up
\end{verbatim}

If only PVCs will be used, they can now be created with a command like

\begin{verbatim}
# atmarp -s 10.0.0.4 0.0.70
\end{verbatim}

NULL encapsulation is used if the \raw{null} keywords is specified.
Note that ARP requires LLC/SNAP encapsulation. NULL encapsulation can
therefore only be used for PVCs.

When using SVCs, some additional configuration work may be necessary. If the
machine is acting as the ATMARP server on that LIS, no additional
configuration is required. Otherwise, the ATM address of the ATMARP
server has to be configured. This is done by creating an entry for the
network address with the option \raw{arpsrv} set, e.g.

\begin{verbatim}
# atmarp -s \
  10.0.0.0 47.0005.80.ffe100.0000.f215.1065.0020EA000756.00 \
  arpsrv
\end{verbatim}

Note that the ATMARP server currently has to be started and configured
before any clients are configured.

The kernel ATMARP table can be read via \path{/proc/net/atm/arp}. The table
used by \name{atmarpd} is regularly printed on standard error if \name{atmarpd}
is started with the \raw{-d} option. If \name{atmarpd} is invoked without
\raw{-d}, the table is written to the file \path{atmarpd.table} in the dump
directory (by default \path{/var/run}; can be changed with \raw{-D}), and
it can be read with \raw{atmarp -a}.


\subsection{LAN Emulation}
\label{lane}

Besides Classical IP over ATM, LAN Emulation (LANE) can be used to
carry IP over ATM. LANE emulates the characteristics of legacy LAN
technology, such as support for broadcasts. LANE server support is
described in \path{atm/lane/USAGE}.

Man pages: \name{bus.8}, \name{lecs.8}, \name{les.8}, and \name{zeppelin.8}

If you plan to run more than one LANE clients, LANE service or LANE
clients and LANE service, you need to specify different local ATM
addresses for each demon. Since all the LANE demons use similar
service access points (SAPs) they need different ATM addresses to
differentiate between connections.

Just as with CLIP, the LANE client consists of two parts: a demon
process called \name{zeppelin} which takes care of the LANE protocol
and kernel part which contains LANE ARP cache.

\name{atmsigd} and \name{ilmid} must already be running when
\name{zeppelin} is started. When \name{zeppelin} starts, the kernel
creates a new interface which can then be configured:

\begin{command}
\# zeppelin \meta{possibly\_more\_options} \&\\
\# ifconfig lec0 \meta{local\_address} \meta{possibly\_more\_options} up\\
\end{command}

In the example below, two LANE clients are started. The first client
uses default interface \raw{lec0}, default listen address and tries to
join the default ELAN. The other LANE client gets interface \raw{lec2}
assigned to it, binds to local address \raw{mybox3}, tries to join
ELAN called \raw{myelan} and will bridge packets between ELAN and
Ethernet segments. Address \raw{mybox3} is defined in
\path{/etc/hosts.atm}. Rest of the bridging can be configured
by reading the Bridging mini-HOWTO.~\cite{bridge-howto}

\begin{verbatim}
# zeppelin &
# ifconfig lec0 10.1.1.42 netmask 255.255.255.0 \
                          broadcast 10.1.1.255 up
#
# zeppelin -i 2 -l mybox3 -n myelan -p &
# ifconfig lec2 10.1.2.42 netmask 255.255.255.0 \
                          broadcast 10.1.2.255 up
\end{verbatim}

By default, \name{zeppelin} uses interface \raw{lec0}, binds to local
ATM address using selector byte value 0, tries to contact LECS using
Well-Known LECS address, joins the default ELAN as defined by the
LECS, accepts the MTU size as defined by the LES and will not act as
an proxy LEC. These parameters can be tailored with command line
options which are defined in \name{zeppelin.8}.

\name{zeppelin} will automatically join any ELANs which use higher
MTU than the default MTU of 1516 bytes. The MTU of the LANE
interface will adjust itself according to the MTU of the current
ELAN.

The state of the LANE ARP cache entries can be monitored through
\path{/proc/net/atm/lec}. For each entry the MAC and ATM addresses and status 
is listed. If the entry has an active connection, the connection
identifiers are also listed.

The LANE service (\name{lecs.8}, \name{les.8} and \name{bus.8}) is
configured using configuration files. The configuration file syntax is
listed on the respective manual pages.

A more detailed description of Linux LANE services is discussed in
Marko Kiiskil\"a's Master's Thesis.~\cite{kiis}


\subsection{MPOA}
\label{mpoa}

The Linux MPOA client continues the tradition of user space -- kernel
divided ATM services. The demon process called \name{mpcd} processes
MPOA control packets while the kernel holds MPOA ingress and egress
caches and does the packet forwarding.

Man page: \name{mpcd.8}

\name{atmsigd} and \name{ilmid} must already be running when
\name{mpcd} is started. Since MPOA detects IP layer flows from LANE
traffic, you need to have \name{zeppelin} running before MPOA can
function. However, the order in which \name{zeppelin} and \name{mpcd}
is started is not fixed. You can kill any of the demons at your will
and restart it later without need to restart the other demon. The
easiest way to disable MPOA is to kill the running \name{mpcd}.

Below is the example from Section~\ref{lane} which starts two LANE
clients. The configuration has been augmented with two MPOA clients
which the LANE clients will serve.

\begin{verbatim}
# zeppelin &
# ifconfig lec0 10.1.1.42 netmask 255.255.255.0 \
                          broadcast 10.1.1.255 up
# mpcd -s mybox1 -l mybox2 &
#
# zeppelin -i 2 -l mybox3 -n myelan -p &
# ifconfig lec2 10.1.2.42 netmask 255.255.255.0 \
                          broadcast 10.1.2.255 up
# mpcd -i 2 -s mybox4 -l mybox5 &
\end{verbatim}

The MPOA demon needs two different local ATM addresses which it uses
when initiating and receiving data and control connections. The
addresses can be the same as with e.g. \name{zeppelin} but must be
different among other \name{mpcd} demons. By default, \name{mpcd} does
not retrieve configuration information from the LECS. The necessary
command line options and an example of using LECS are shown on the
\name{mpcd} manual page.  The manual page also lists the rest of the
available options.


The contents of MPOA ingress and egress caches can be monitored
through \path{/proc/net/atm/mpc} file.

The Linux MPOA client also supports CBR traffic class for shortcuts
SVCs instead of default UBR. The QoS specifications for future
shortcuts can be set and modified using \path{/proc/net/atm/mpc}.

\begin{verbatim}
# echo add 130.230.54.146 tx=80000,1600 rx=tx > /proc/net/atm/mpc
#             # generate enough traffic to trigger a shortcut
# cat /proc/net/atm/mpc 
QoS entries for shortcuts:
IP address
  TX:max_pcr pcr     min_pcr max_cdv max_sdu
  RX:max_pcr pcr     min_pcr max_cdv max_sdu
130.230.54.146  
     80000   0       0       0       1600   
     80000   0       0       0       1600   

Interface 2:

Ingress Entries:
IP address      State     Holding time  Packets fwded  VPI VCI
130.230.4.3     invalid   1160          0           
130.230.54.146  resolved  542           151            0   109
...
\end{verbatim}

The shortcut to IP address \raw{130.230.54.146} was established with
the parameters shown above. There also exist patches which extend the
flow detection to fully support layer 4 flows. The layer 4 flows are
expressed as a 5 tuple (proto, local addr, local port, remote addr,
remote port) and they identify application to application flows. If
you are interested, see
\url{ftp://sunsite.tut.fi/pub/Local/linux-atm/mpoa/} for the latest
patch.

\begin{thebibliography}{8}
%  \bibitem{I361}ITU-T Recommendation I.361.
%    {\em B-ISDN ATM layer specification},
%    ITU, 1993.
  \bibitem{api}Almesberger, Werner.
    {\em Linux ATM API},
    \url{ftp://icaftp.epfl.ch/pub/linux/atm/api/},
    July 1996.
%  \bibitem{RFC1483}Heinanen, Juha.
%    {\em Multiprotocol Encapsulation over ATM Adaptation Layer 5},
%    RFC1483, July 1993.
  \bibitem{RFC1577}Laubach, Mark.
    {\em Classical IP and ARP over ATM},
    RFC1577, January 1994.
%  \bibitem{RFC1755}RFC1755; Perez, Maryann; Liaw, Fong-Ching; Mankin, Allison;
%    Hoffman, Eric; Grossman, Dan; Malis, Andrew.
%    {\em ATM Signaling Support for IP over ATM},
%    IETF, 1995.
% reference RFC
   \bibitem{bridge-howto}Cole, Christopher.
     {\em Bridging mini-Howto},
     \url{http://metalab.unc.edu/LDP/HOWTO/mini/Bridge.html},
     September 1998.
   \bibitem{kiis} Kiiskil\"a, Marko.
     {\em Implementation of LAN Emulation Over ATM in Linux},
     \url{ftp://sunsite.tut.fi/pub/Local/linux-atm/misc/},
     October 1996.
   \bibitem{lanev1} ATM Forum.
     {\em LAN Emulation Over ATM -- Version 1.0},
     February 1996.
   \bibitem{lanev2} ATM Forum.
     {\em LAN Emulation Over ATM -- Version 2 -- LUNI Specification},
     July 1997.
   \bibitem{mpoav1} ATM Forum.
     {\em Multi-Protocol Over ATM -- Version 1.0},
     July 1997.
\end{thebibliography}

%%beginskip
\end{document}
%%endskip
